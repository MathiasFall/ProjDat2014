\documentclass[11pt]{article}
\usepackage{geometry} % See geometry.pdf to learn the layout options. There are lots.
\geometry{letterpaper} % ... or a4paper or a5paper or ...
%\geometry{landscape} % Activate for for rotated page geometry
\usepackage[parfill]{parskip} % Activate to begin paragraphs with an empty line rather than an indent
\usepackage{caption, subcaption} % enable subfigures
\usepackage{listings}
\renewcommand*{\refname}{\section{Referencer}}
\usepackage[stable]{footmisc}
\usepackage{graphicx}
\usepackage{amssymb}
\usepackage{amsmath}
\usepackage[usenames,dvipsnames]{color}
\usepackage[ruled,vlined,lined, linesnumbered]{algorithm2e}
\usepackage{float}
\usepackage[utf8]{inputenc}
\usepackage{epstopdf}
\renewcommand*\contentsname{Indhold}
\renewcommand{\figurename}{Fig.}
\DeclareGraphicsRule{.tif}{png}{.png}{`convert #1 `dirname #1`/`basename #1 .tif`.png}
\graphicspath{ {/home/fall/Desktop/} }
% Farver til javakode
\definecolor{dkgreen}{rgb}{0,0.6,0}
\definecolor{gray}{rgb}{0.5,0.5,0.5}
\definecolor{mauve}{rgb}{0.58,0,0.82}

% Indstillinger til kodeinput
\lstset{literate=
  {á}{{\'a}}1 {é}{{\'e}}1 {í}{{\'i}}1 {ó}{{\'o}}1 {ú}{{\'u}}1
  {Á}{{\'A}}1 {É}{{\'E}}1 {Í}{{\'I}}1 {Ó}{{\'O}}1 {Ú}{{\'U}}1
  {à}{{\`a}}1 {è}{{\`e}}1 {ì}{{\`i}}1 {ò}{{\`o}}1 {ù}{{\`u}}1
  {À}{{\`A}}1 {È}{{\'E}}1 {Ì}{{\`I}}1 {Ò}{{\`O}}1 {Ù}{{\`U}}1
  {ä}{{\"a}}1 {ë}{{\"e}}1 {ï}{{\"i}}1 {ö}{{\"o}}1 {ü}{{\"u}}1
  {Ä}{{\"A}}1 {Ë}{{\"E}}1 {Ï}{{\"I}}1 {Ö}{{\"O}}1 {Ü}{{\"U}}1
  {â}{{\^a}}1 {ê}{{\^e}}1 {î}{{\^i}}1 {ô}{{\^o}}1 {û}{{\^u}}1
  {Â}{{\^A}}1 {Ê}{{\^E}}1 {Î}{{\^I}}1 {Ô}{{\^O}}1 {Û}{{\^U}}1
  {œ}{{\oe}}1 {Œ}{{\OE}}1 {æ}{{\ae}}1 {Æ}{{\AE}}1 {ß}{{\ss}}1
  {ç}{{\c c}}1 {Ç}{{\c C}}1 {ø}{{\o}}1 {å}{{\r a}}1 {Å}{{\r A}}1
  {€}{{\EUR}}1 {£}{{\pounds}}1,
  frame=single,
  language=Java,
  aboveskip=3mm,
  belowskip=3mm,
  showstringspaces=false,
  columns=flexible,
  basicstyle={\small\ttfamily},
  numbers=left,
  numberstyle=\tiny\color{gray},
  keywordstyle=\color{blue},
  commentstyle=\color{dkgreen},
  stringstyle=\color{mauve},
  breaklines=true,
  breakatwhitespace=true
  tabsize=3}

%define paper title, author and date
%if date is empty, defaults to current date
\title{Automatisk tabeloprettelse og fejltjek til forsøgsresultater på Rigshospitalet\\
3. Delrapport - ProjDat 2014}
\author{Instruktor: Markus Lund Wittorf \\Gruppemedlemmer: \\ Mikkel Aleksander Høgh Rasmusen - 100888 \\ Mathias Fall Christensen - 020693 \\ Markus Visvaldis Ingemann Thieden - 170594}
\date{}

\begin{document}
\maketitle
\pagebreak
\tableofcontents
\pagebreak
\section{Litteraturreview}

\pagebreak
\section{Delrapport}
\subsection{Abstract}
Vi laver en desktopapplikation til automatisk oprettelse af tabeller med henblik på forsøgs- resultatindsamling for Professor Lars Rasmussen fra anæstesi afd. på Rigshospitalet. Målet med projektet er en simplificere processen fra indledende forsøgsprotokol til klargjort tabel. Dette vil vi gøre ved at analysere protokollen, og automatisk oprette et dokument med de i protokollen angivne parametre. Outputtet er et dokument der kan læses af excel eller lignende programmer, som fungerer som en skabelon til det specifikke forsøg. Derudover har vi en fejltjek-funktion, der kan undersøge om et udfyldt skema overholder specifikationerne.


\subsection{IT-projektets formål og rammer}
\subsubsection{Functionality}
Opretter automatisk skemaer til forsøgsresultater ud fra protokoller. Tjekker indtastede værdier for varians fra angivne grænseværdier.
\subsubsection{Application domain}
Forskere på rigshospitalet, evt. sekretærer der står for dataindtastning.
\subsubsection{Conditions}
Programmet udvikles som led i projektkurset. Programmet vil blive anvendt til at reducere indtastningsfejl i forbindelse med forskningsresultater.
\subsubsection{Technology}
Programmet skrives i Java. Skrives til at være kompatibelt med rigshospitalets Windows-systemer.
\subsubsection{Objects}
Protokoller, skemaer og CRF'er (Case Report Files).
\subsubsection{Responsibility}
Automatisering af oprettelse af skemaer. Fejltjekning af indtastet data.

\subsection{Kravspecifikation for IT-løsningen}
De funktionelle krav udgør specifikationer af funktioner, som programmet skal understøtte. Ikke funktionelle krav er kvalitative begrænsninger, som indirekte er relateret til programmets funktionalitet. \footnote{Se evt. s. 12-13 i OOSE\cite{OOSE}}
\subsubsection{Funktionelle krav}
\begin{itemize}
  \item Skal i praksis kunne håndtere et stort antal parametre til oprettelse af tabeller.
  \item Skal kunne oprette et dokument med forsøgsparametre, som kan læses af Excel.
  \item Skal kunne "oversætte" en forsøgsprotokol og opfange parametre samt begrænsninger.
  \item Skal kunne sammenligne et udfyldt excelskema med forsøgsprotokollens krav, og advare brugeren hvis de ikke er overholdt.
\end{itemize}
\subsubsection{Ikke-funktionelle krav}
\begin{itemize}
  \item Skal kunne oprette tabeller hurtigere end manuel indtastning.
  \item Simpelt og overskueligt design.
  \item Skal være brugervenligt.
\end{itemize}
\pagebreak
\subsubsection{Use-case model}
\begin{figure}[H]
\includegraphics[scale=0.5]{usecase.png}
\caption{Use-case model}
\end{figure}
Forskeren kører Writer funktionen for at oprette et Excel ark,
eller Fejltjek for at tjekke et skema.
Analyse indsamler information fra Protokollen.
Writer tager input fra Forskeren, henter information fra Analyse, og skriver Excel arket. Fejltjek modtager parametre fra Forskeren, og sammenholder informationen fra Analyse med skemaet angivet af Forskeren.
\subsubsection{Specificerede use-cases}
I opgavebeskrivelsen står der, at vi skal udforme 3 use-cases, men da vores program i realiteten kun skal kunne de ting, vi har beskrevet nedenfor, har det kun været muligt for os at lave disse to:\\
\begin{itemize}
\item[\textbf{USE CASE 1:}] Opret Excel Ark\\
\noindent\rule{14cm}{0.4pt}

\item [Participating actors:] \quad Forsker \\
\noindent\rule{14cm}{0.4pt}

\item [Flow of events:]
\begin{enumerate}
\item Forsker vælger stien for Protokol.
\item Forsker vælger stien for skema output.
\item Forsker indlæser protokollen.
\item Programmet læser værdier fra protokollen og viser dem i brugergrænsefladen
\item Forsker gennemlæser og bekræfter de indlæste værdier i brugergrænsefladen
\item Forsker opretter skemaet ved tryk på "Opret Skema"
\item Programmet skriver værdierne til en excel fil på den i skridt (2) valgte sti.
\end{enumerate}
\noindent\rule{14cm}{0.4pt}
\item[Entry condition:] Forskeren har startet programmet.\\
\noindent\rule{14cm}{0.4pt}

\item[Exit condition:]
\begin{itemize}
  \item Skemaet er blevet oprettet.
  \item Programmet afsluttes.
\end{itemize}
\noindent\rule{14cm}{0.4pt}
\item[Quality requirements:]
\begin{itemize}
\item Programmet skal oprette skemaet hurtigere end Forsker ville kunne gøre det på egen hånd.
\item Programmet skal finde alle relevante attributter og grænseværdier.
\end{itemize}

\noindent\rule{14cm}{0.4pt} \hfill \\

\end{itemize}
\begin{itemize}
\item[\textbf{USE CASE 2:}] Fejltjek Excel Ark\\
\noindent\rule{14cm}{0.4pt}
\item [Participating actors:] Forsker\\
\noindent\rule{14cm}{0.4pt}

\item [Flow of events:]
\begin{enumerate}
\item Forsker vælger stien for Protokol.
\item Forsker vælger stien for Excel Ark.
\item Forsker starter fejltjek ved tryk på "Kør tjek".
\item Programmet læser grænseværdier fra protokollen og sammenholder dem med værdierne i Excel arket.
\item Forsker aflæser fejlagtige værdier hvis de er tilstede, og foretager eventuelle rettelser.
\end{enumerate}
\item [Entry condition:] Forsker vælger "Tjek Skema" tabben.\\
\noindent\rule{14cm}{0.4pt}

\item [Exit condition:]
\begin{itemize}
\item Tjekket er kørt igennem.
\item Programmet afsluttes.
\end{itemize}
\noindent\rule{14cm}{0.4pt}

\item [Quality requirements:]
\begin{itemize}
\item Programmet skal vise en oversigt over alle værdier, der ikke ligger indenfor de i protokollen
angivne grænseværdier. \\
\end{itemize}
\noindent\rule{14cm}{0.4pt}

\end{itemize}

\subsubsection{Klassediagram}
\begin{figure}[h!]
\includegraphics[scale=0.5]{Klassediagram1.png}
\caption{Klassediagram}
\end{figure}
Forskeren bruger GUI'en til til at køre enten Write eller Fejltjek modulerne. Både Write og Fejltjek modulerne afhænger af analysedataen, som bliver lavet af Analysemodulet ud fra protokollen vha. PDFbox. PDFbox udtrækker tekst fra protokollen (som ligger i .pdf format), og vi kører regex på de relevante dele af denne tekst. Write modulet skriver de udtrukne data (vha. JExcel Write) til et excel dokument. Fejltjek modulet læser et excel dokument vha. JExcel Read. Den læste data sammenholdes med den fra protokollen udtrukne tekst, sammenligner værdierne (tjekker om de ligger indenfor de acceptable værdier) og sender resultaterne videre til GUI'en, som viser dem til forskerne/brugerne.

\subsubsection{Sekvens-diagrammer}
Vi har lavet to sekvensdiagrammer, der passer til de to use-cases vi har beskrevet tidligere.
\begin{figure}[H]
\includegraphics[scale=0.4]{Sekvensdiagram1v3} \hfill \\\\
\caption{Sekvensdiagram for Use-case 1}
\end{figure}
Fig. 3 er et sekvensdiagram over den første use-case, hvor forskeren benytter programmet til at oprette et skema ud fra en protokol. Brugeren angiver stien for protokollen og den ønskede placering for Excel-arket (se fig. 5 for mockup). Et Excel-ark oprettes på den angivne output-sti, og protokolfilen sendes til PDFbox librariet, der oversætter det til et læsbart (for programmet) format. Analysemodulet kører regex på denne data, og finder de relevante parametre. Disse bliver sendt til JExcel, som skriver den fundne data ind i Excel-arket. Til sidst vises det færdige Excel-ark for brugeren.\\
\begin{figure}[H]
\includegraphics[scale=0.4]{Sekvensdiagram2v3}
\caption{Sekvensdiagram for Use-case 2}
\end{figure}
Fig. 4 er et sekvensdiagram over den anden use-case, hvor forskeren benytter sig af den anden funktionalitet i programet til at tjekke et skema igennem for fejl (se fig. 6 for mockup). Brugeren angiver stierne for både protokollen og Excel-arket. Excel-arket sendes til JExcel, som læser dataen fra de af brugeren indtastede felter. Denne data sendes så til analysemodulet. Protokollen sendes til PDFbox, som ligesom i fig. 3 læser filen og sender dataen til analysemodulet. Analysemodulet sammenligner dataen fra PDFbox og JExcel, og sender resultatet til GUI'et, som viser det for brugeren.
\subsection{Systemdesign sammenfatning}
\subsubsection{System-design resume}
Da vi laver et program der skal kunne oprette excel dokumenter og læse PDF filer, har vi benyttet os af to ikke-standard Javapakker; PDFbox og JExcel\footnote{Kilderne kan findes på hhv. http://pdfbox.apache.org/ og http://jexcelapi.sourceforge.net/}. Metoderne i disse klasse danner stort grundlag for vores program, og vores opgave ligger i at bearbejde den data som PDFbox kan trække ud for os. Vi har 3 hoved moduler i programmet; \begin{itemize}

\item Analyse, der benytter sig af PDFbox
\item Writer, der benytter sig af JExcel til at oprette et excel dokument
\item Fejltjek, der benytter sig af begge pakker til at læse en protokol og et excel skema.
\end{itemize}
De sidstnævnte moduler er pt. ikke implementeret.
Analyse klassen er dog udarbejdet i en grov form (se punkt 2.7).
\subsubsection{Udestående design- og implementationsopgaver}
Fejltjek og Writer klasserne er ikke blevet implementeren endnu, og det samme gælder GUI'en. Mht. til GUI'en, har vi dog en god idé om, hvordan den skal se ud jf. vores mockups. Fejltjek og Write afhænger i stor grad af Analyse klassen, så vi arbejder meget linært - vi kan ikke begynde på det næste modul, før den første er implementeret.\\I Analyse klassen, mangler vi en del der kan bearbejde den data der bliver udtrukket fra PDF'en. Dette har vi umidelbart tænkt os at gøre vha. Scanner klassen fra Javas standard bibliotek.

\subsection{Program- og systemtest}
Vores tests består hovedsageligt af at køre de enkelte moduler, for at tjekke deres funktionalitet. Vi har planer om at fremvise en prototype af produktet til vores kunde. En kørsel af Analyse klassen er at finde under punkt 2.7.

\subsection{Brugergrænseflade og interaktionsdesign}
\begin{figure}[H]
\includegraphics[scale=1]{osm.png}
\caption{Mockup af brugergrænseflade mht. Use-case 1}
\end{figure}
Her ses et mockup af vores brugergrænseflade. Protokolstien kan angives, og brugeren kan specificere hvor excelfilen skal oprettes. Tabellen viser en oversigt over de fundne attributter fra protokollen.\\
Som det ses, kan der navigeres mellem programmets to dele via fanerne i toppen af vinduet.
\begin{figure}[H]
\includegraphics[scale=0.8]{mockup2.png}
\caption{Mockup af brugergrænseflade mht. Use-case 2}
\end{figure}

Dette er et mockup af den anden del af programmet. Her er idéen at der er blevet angivet et excel skema og den originale protokol, hvorefter der er blevet kørt et fejltjek. Tabellen viser evt. fejl eller afvigelser fra standard syntaks (f.eks. er der under køn blevet skrevet "Mand" i celle C-4, men protokollen har bedt om et input i form af M/K).
\pagebreak
\subsection{Versionsstyring}
På nuværende tidspunkt, har vi implementeret en grov version af klassen Analyse, der kan læse en PDF, og trække relevant data ud.
Da vi pt ikke benytter os af github, kan vi ikke vise en commit-log, men vi har i sinde at begynde på det. I stedet har vi nedenfor den første implementation af Analyse-klassen, der giver et godt indblik i, hvad den skal gøre:

\lstinputlisting[language=java]{Analyse.java}

En kørsel af programmet med en protokol-PDF, vil finde og udsrive alle linjer omsluttet med "//...//", og som overholder en bestemt syntaks (e.g. //;Køn;Alder;Blodtype//). Herefter skal en anden del af klassen kunne separere attributterne (f.eks. hva. Scanner klassen fra Javas standard bibliotek) \\Den overordnede idé er, at det skal kunne være muligt at "browse" gennem fil stier via GUI'en, for at finde imput filen. Dette vil først være muligt når GUI'en er implementeret.

\subsubsection{Vigtigste ændringer}
Siden vores sidste aflevering, har vi ikke lavet nogle betydelige ændringer i vores kode. Vi forventer at have en endelig version af Analyse klassen, samt en grov implementation af Writer klassen klar indenfor den næste uge.
\subsection{Projektsamarbejdet}
Samarbejdet fungerer fint, så vi ser ingen grund til at ændre i dette. Vi er gode til at arbejdsfordele, og vores forskellige evner og styrker komplimenterer hinanden udmærket.\\
Vi har fortsat tænkt os at holde møder på ugentlig basis, hvor kodning og evt. rettelser vil finde sted. Dog skal det nævnes at vi, på opfordring fra vores instruktor (og fordi det er et krav til opgaven) har tænkt os at begynde på at bruge github (se mere i afsnit 2.8.3).\\
\subsubsection{Hvad går godt?}
Vi er afklarede om målet for vores projekt.\\
Fremmødet har været upåklageligt.\\
Arbejdsmoralen er høj.\\
Samarbejdet fungerer stadig godt.
\subsubsection{Hvad går mindre godt?}
Vi er først lige nu ved at komme i gang med github.\\
Tidspresset er begyndt at tage til, hvilket kan have en negativ indflydelse på arbejdsmoralen. Vi har dog stadig et optimistisk syn på projektet, og sporadiske Cafeen? besøg holder humøret oppe.
\subsubsection{Effektivisering af udviklingsarbejde}
Vi vil begynde at bruge github, så vi kan arbejde uafhængigt af hinanden. Dette er ikke ensbetydende med at vi vil begynde at arbejde individuelt hele tiden, men hvis vi kan komme undgå at skulle mødes hver gang, vi skal kode, kan de i høj grad få øget tempoet på processen.
\begin{thebibliography}{9}

\bibitem{OOSE}
  Object-Oriented Software Engineering Using UML, Patters, and Java.
  \emph{Bernd Bruegge, Allen H. Dutoit}
  3rd Edition,
  2014.


\end{thebibliography}

\end{document}